%author: Ondrej Lukasek (xlukas15)

\documentclass[11pt, a4paper, twocolumn]{article}
\usepackage[left=1.4cm,text={18.2cm, 25.2cm},top=2.3cm]{geometry}
\usepackage[IL2]{fontenc}
\usepackage[utf8]{inputenc}
\usepackage{setspace}
\usepackage[czech]{babel}
\usepackage{parskip}
\usepackage{amsthm, amssymb, amsmath}
\usepackage{times}
\usepackage{hyperref}
\usepackage{siunitx}
\usepackage{dsfont}

\newtheorem{theorem}{Theorem}

\hypersetup{hidelinks}
\setlength{\parindent}{\baselineskip}
\setlength{\parskip}{0em}
\theoremstyle{definition}
\newtheorem{definice}{Definice}
\newtheorem{veta}{Věta}

\begin{document}

\begin{titlepage}
\begin{center}
\Huge \textsc{Vysoké učení technické v~Brně}

\huge \textsc{Fakulta informačních technologií}\\
\vspace{\stretch{0.382}}
{\LARGE Typografie a~publikování -- 2. projekt\\
Sazba dokumentů a~matematických výrazů\\}
\vspace{\stretch{0.618}}
\end{center}
{\Large 2023 \hfill Ondřej Lukášek (xlukas15)}
\end{titlepage}

\section*{Úvod}

V této úloze si vyzkoušíme sazbu titulní strany, ma\-te\-ma\-tic\-kých vzorců, prostředí a dalších textových struktur obvyklých pro technicky zaměřené texty\,--\,například De\-fi\-ni\-ce~\ref{definice1} nebo rovnice~\eqref{rovnice3} na straně~\pageref{rovnice3}. Pro vytvoření těchto odkazů používáme kombinace příkazů \verb|\label|, \verb|\ref|, \verb|\eqref|~a~\verb|\pageref|. Před odkazy patří nezlomitelná mezera. Pro zvýrazňování textu jsou zde několikrát po\-u\-ži\-ty příkazy \verb|\verb| a \verb|\emph|.\par
Na titulní straně je použito prostředí \verb|titlepage| a sá\-ze\-ní nadpisu podle optického středu s využitím \emph{přesného} zlatého řezu. Tento postup byl probírán na přednášce. Dále jsou na titulní straně použity čtyři různé velikosti písma a mezi dvojicemi řádků textu je použito odřádkování se zadanou relativní velikostí 0,5\,em a 0,4\,em\footnote{Nezapomeňte použít správný typ mezery mezi číslem a jednotkou.}.

\section{Matematický text}

% mezera u ${}$ mezi zavorkami
V této sekci se podíváme na sázení matematických sym\-bo\-lů a výrazů v plynulém textu pomocí prostředí \verb|math|. Definice a věty sázíme pomocí příkazu \verb|\newtheorem| s~využitím balíku \verb|amsthm|. Někdy je vhodné použít kon\-struk\-ci \verb|${}$| nebo \verb|\mbox{}|, která říká, že (matematický) text nemá být zalomen.

\medskip

\begin{definice} \label{definice1}
Zásobníkový automat \emph{(ZA) je definován jako sedmice tvaru $A = (Q, \Sigma, \Gamma, \delta, q_0, Z_0, F)$, kde:} 

\medskip

\begin{itemize} \setlength \itemsep{1em} % s timhle si pohrat
    \item \emph{$Q$ je konečná množina} vnitřních (řídicích) stavů,
    \item \emph{$\Sigma$ je konečná} vstupní abeceda,
    \item \emph{$\Gamma$ je konečná} zásobníková abeceda,
    \item \emph{$\delta$ je} přechodová funkce $Q \times (\Sigma\cup\{\epsilon\}) \times \Gamma \rightarrow 2^{Q \times \Gamma^*}$,
    \item \emph{$q_0 \in Q$ je} počáteční stav, \emph{$Z_0 \in \Gamma$ je} startovací symbol zásobníku \emph{a $F \subseteq Q$ je množina} koncových stavů.
\end{itemize}

\medskip

Nechť $P = (Q, \Sigma, \Gamma, \delta, q_0, Z_0, F)$ je ZA. \emph{Konfigurací} nazveme trojici $(q, w, \alpha) \in Q \times \Sigma^* \times \Gamma^*$, kde $q$ je aktuální stav vnitřního řízení, $w$ je dosud nezpracovaná část vstup\-ní\-ho řetězce a $\alpha = Z_{i_1} Z_{i_2} \dots Z_{i_k}$ je obsah zásobníku.
\end{definice}

\subsection{Podsekce obsahující definici a větu}

\begin{definice}
    Řetězec $w$ nad abecedou $\Sigma$ je přijat ZA \emph{$A$ jest\-li\-že $(q_0, w, Z_0) \overset{\ast}{\underset{A}{\vdash}} (q_F, \epsilon, \gamma)$ pro nějaké $\gamma \in \Gamma^*$ a $q_F \in F$. Množina $L(A) = \{w \mid w$ je přijat ZA $A\} \subseteq \Sigma^*$ je} jazyk přijímaný ZA $A$.
\end{definice}

\medskip

\begin{veta}
    \emph{Třída jazyků, které jsou přijímány ZA, odpovídá} bezkontextovým jazykům.
\end{veta}

\section{Rovnice}
Složitější matematické formulace sázíme mimo plynulý text pomocí prostředí \verb|displaymath|. Lze umístit i ně\-ko\-lik výrazů na jeden řádek, ale pak je třeba tyto vhodně oddělit, například příkazem \verb|\quad|.

\begin{displaymath}
    1^{2^3} \neq \Delta^1_{\Delta^2_{\Delta^3}} \quad y^{11}_{22} - \sqrt[9]{x + \sqrt[7]{y}} \quad x > y_1 \leq y^2
\end{displaymath}

\noindent V rovnici~\eqref{rovnice2} jsou využity tři typy závorek s různou \emph{ex\-pli\-cit\-ně} definovanou velikostí. Také nepřehlédněte, že nasledující tři rovnice mají zarovnaná rovnítka, a použijte k~tomuto~účelu vhodné prostředí.

% MOC VELKA MEZERA

\begin{eqnarray}
    -\cos^2{\beta} & = & \frac{\frac{\frac{1}{x} + \frac{1}{3}}{y} + {1000}}{\overset{8}{\underset{j = 2}{\prod}}q_j} \\
    \label{rovnice2}
    \bigg(\Big\{b\, \star\, \bigr[3 \div 4\bigr] \circ a\Big\}^{\frac{2}{3}}\bigg) & = & \log_{10}x \\
    \label{rovnice3}
    \int^b_a f(x)\,dx & = & \int^d_c f(y)\,dy
\end{eqnarray}

\noindent V této větě vidíme, jak vypadá implicitní vysázení limity $\lim_{m \to \infty} f(m)$ v normálním odstavci textu. Podobně je to~i~s~dalšími symboly jako $\bigcup_{N \in \mathcal{M}}N$ či $\sum_{i=1}^m x_i^2$. S vy\-nu\-ce\-ním méně úsporné sazby příkazem \verb|\limits| budou vzorce vysázeny v podobě $\lim\limits_{m \to \infty}f(m)$ a $\sum\limits^m_{i=1}x^4_i$.

\section{Matice}
Pro sázení matic se velmi často používá prostředí \verb|array| a závorky (\verb|\left|, \verb|\right|).

% MOC VELKA MEZERA

$$ \text{\textbf{B}} = \left|
\begin{array}{cccc}
b_{11} & b_{12} & \cdots & b_{1n} \\
b_{21} & b_{22} & \cdots & b_{2n} \\
\vdots & \vdots & \ddots & \vdots \\
b_{m1} & b_{m2} & \cdots & b_{mn}
\end{array} \right|
= \left|
\begin{array}{cc}
t & u\\
v & w
\end{array}\right|
= tw - uv$$

$$ \mathbb{X} = \textbf{Y} \iff \bigg[
\begin{array}{ccc}
     & \Omega + \Delta & \hat{\psi} \\
    \vec{\pi} & \omega & 
\end{array}
\bigg] \neq 42
$$

Prostředí \verb|array| lze úspěšně využít i jinde, například~na pravé straně následující rovnice. Kombinační číslo na levé straně vysázejte pomocí příkazu \verb|\binom|.

$$
\binom{n}{k} = \left\{
\begin{array}{c l}
     0 & \text{pro } k<0 \\
     \frac{n!}{k!(n-k)!} & \text{pro } 0 \leq k \leq n \\
     0 & \text{pro } k>0
\end{array} \right. $$

\end{document}
